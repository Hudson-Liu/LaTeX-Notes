\documentclass{article}
\usepackage[left=3cm,right=3cm,top=0cm,bottom=2cm]{geometry}
\usepackage{amssymb}
\usepackage{amsmath}
\usepackage{physics}
\usepackage{mathtools}

\setlength{\parindent}{0mm}
\DeclarePairedDelimiter{\opair}{\langle}{\rangle}

\title{Multivar. Class Notes}
\author{Hudson Liu}
\date{Sep. 9, 2024}

\newcommand{\R}{\mathbb{R}}

\begin{document}
\maketitle
Finding a Dot Product:

\[u=\opair{5, -1, 0}\]
\[v=\opair{-2, -3, 4}\]
\[u * v = 5(-2) + (-1)(-3) + 0(4)\]

Three Types of Scalar Multiplication:
\begin{itemize}
  \item Scalar Multiplication
  \item Dot Product
  \item Cross Product (Non-Commutitative)
\end{itemize}

Properties of Dot Product:
\begin{itemize}
  \item Communitative
  \item Distributative
  \item NOT Associative
  \item Associative w/ Scalar Multiplication
  \item Multiplying by a Zero-Vector Results in Zero-Vector
  \item $v * v = \norm{v}^2$
\end{itemize}

Dot product is very "cooperative"; abides by the properties that we want.

\[
  \cos(\theta)=\frac{ u*v }{ \norm{v} \norm{w} }
\]

Note that $\norm{u}$ and $\norm{v}$ will always be $>$ 0. Thus, $\cos(\theta)$ btwn. two vector will always have the same sign as u * v.

\[
  [0, \frac{\pi}{2}], \text{cos is positive.}
\]
\[
  [\frac{\pi}{2}, \pi], \text{cos is negative.}
\]

Thus:
\begin{itemize}
  \item Positive dot-product $\rightarrow$ Acute angle
  \item Zero dot-product $\rightarrow$ Right angle
  \item Negative dot-product $\rightarrow$ Obtuse angle
\end{itemize}

Orthogonal: Essentially just "right angle."
\begin{itemize}
  \item Perpendicular, Orthogonal, Normal; All Synonyms
  \item Orthogonal is used mostly with vectors.
\end{itemize}

\[
  u * v = \norm{u} \norm{v} \cos(\theta)
\]

Projections: 

Two vectors, u \& v. Project u onto v; form a right angle and find the perpendicular "shadow" of u on v. Think abt it like breaking u up into components that are perpendicular and parallel to some other vector v. Also referred to as vector decomposition.

$ (proj_v)(u) $

we'll name the other component the "orthogonal" component\dots
\[ (orth_v)(u) \]
\[ (proj_v)(u) = (\frac{u \cdot v}{\norm{v}^2}) * v \]

To find the orthogonal projection, just subtract the projection from u

\end{document}
