\documentclass{article}
\usepackage[left=3cm,right=3cm,top=0cm,bottom=2cm]{geometry}
\usepackage{amssymb}
\usepackage{amsmath}
\usepackage{physics}
\usepackage{mathtools}

\setlength{\parindent}{0mm}
\DeclarePairedDelimiter{\opair}{\langle}{\rangle}

\title{Diff. Eq. Notes}
\author{Hudson Liu}
\date{Sep. 23, 2024}

\newcommand{\R}{\mathbb{R}}

\begin{document}
\maketitle
Cross-Product:
\begin{itemize}
    \item Big dijefiwoajof. btwn. this and dot product is that dot product $\rightarrow$ scalar, cross product $\rightarrow$ vector
    \item We often need to find a vector in $\R^3$ that is orthogonal to 2 other vectors.
    \item Note that these two vectors do not need to be orthogonal to one another.
\end{itemize}
\[
    \vec{u} = \opair{u_1, u_2, u_3}, \vec{v} = \opair{v_1, v_2, v_3}
\]
\[
\begin{vmatrix}
    i & j & k\\
    u_1 & u_2 & u_3\\
    v_1 & v_2 & v_3
\end{vmatrix}
\]
Helpful Product-like Properties
\begin{itemize}
    \item Distributive: $u \times (v + w) = u \times v + u \times w$
    \item Scalar Associative $c(\vec{u} \times \vec{v})= c\vec{u} \times \vec{v} = \vec{u} \times c\vec{v}$
    \item $\vec{v} \times \vec{0} = \vec{0}$
\end{itemize}

Less Intuitive Properties
\begin{itemize}
    \item Anti-Communitative $\vec{u} \times \vec{v} = -(\vec{v} \times \vec{u})$
    \item $\vec{v} \times \vec{v} = \vec{0}$
    \item $\vec{u} \cdot (\vec{v} \times \vec{w}) = (\vec{u} \times \vec{v}) \cdot \vec{w}$
\end{itemize}
Note that both 3 and 5 essentially say that the cross product only returns the zero vector when the two vectors are parallel.\newline

\textbf{Geometric Properties}
\[
    \norm{\vec{u} \times \vec{u}} = \norm{\vec{u}} \cdot \norm{\vec{v}} \sin{\theta}
\]
\[
    \abs{\vec{u} \cdot (\vec{v} \times \vec{w})} \text{ is the volume of the parallelepiped formed by u, v, and w}
\]
Any two vectors in $\R^2$ form a parallelegram. The same applies to $\R^3$, thus creating a parallelopiped. The two-dimensional "cross product" can be considered the area of this parallelogram in $\R^2$.
\end{document}
