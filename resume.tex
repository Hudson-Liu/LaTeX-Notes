\documentclass[letterpaper,11pt]{article}

\usepackage{latexsym}
\usepackage[empty]{fullpage}
\usepackage{titlesec}
\usepackage{marvosym}
\usepackage[usenames,dvipsnames]{color}
\usepackage{verbatim}
\usepackage{enumitem}
\usepackage[hidelinks]{hyperref}
\usepackage{fancyhdr}
\usepackage[english]{babel}
\usepackage{tabularx}
\input{glyphtounicode}


%----------FONT OPTIONS----------
% sans-serif
% \usepackage[sfdefault]{FiraSans}
% \usepackage[sfdefault]{roboto}
% \usepackage[sfdefault]{noto-sans}
% \usepackage[default]{sourcesanspro}

% serif
% \usepackage{CormorantGaramond}
% \usepackage{charter}


\pagestyle{fancy}
\fancyhf{} % clear all header and footer fields
\fancyfoot{}
\renewcommand{\headrulewidth}{0pt}
\renewcommand{\footrulewidth}{0pt}

% Adjust margins
\addtolength{\oddsidemargin}{-0.5in}
\addtolength{\evensidemargin}{-0.5in}
\addtolength{\textwidth}{1in}
\addtolength{\topmargin}{-.5in}
\addtolength{\textheight}{1.0in}

\urlstyle{same}

\raggedbottom
\raggedright
\setlength{\tabcolsep}{0in}

% Sections formatting
\titleformat{\section}{
  \vspace{-4pt}\scshape\raggedright\large
}{}{0em}{}[\color{black}\titlerule \vspace{-5pt}]

% Ensure that generate pdf is machine readable/ATS parsable
\pdfgentounicode=1

%-------------------------
% Custom commands
\newcommand{\resumeItem}[1]{
  \item\small{
    {#1 \vspace{-2pt}}
  }
}

\newcommand{\resumeSubheading}[4]{
  \vspace{-2pt}\item
    \begin{tabular*}{0.97\textwidth}[t]{l@{\extracolsep{\fill}}r}
      \textbf{#1} & #2 \\
      \textit{\small#3} & \textit{\small #4} \\
    \end{tabular*}\vspace{-7pt}
}

\newcommand{\resumeSubSubheading}[2]{
    \item
    \begin{tabular*}{0.97\textwidth}{l@{\extracolsep{\fill}}r}
      \textit{\small#1} & \textit{\small #2} \\
    \end{tabular*}\vspace{-7pt}
}

\newcommand{\resumeProjectHeading}[2]{
    \item
    \begin{tabular*}{0.97\textwidth}{l@{\extracolsep{\fill}}r}
      \small#1 & #2 \\
    \end{tabular*}\vspace{-7pt}
}

\newcommand{\resumeSubItem}[1]{\resumeItem{#1}\vspace{-4pt}}

\renewcommand\labelitemii{$\vcenter{\hbox{\tiny$\bullet$}}$}

\newcommand{\resumeSubHeadingListStart}{\begin{itemize}[leftmargin=0.15in, label={}]}
\newcommand{\resumeSubHeadingListEnd}{\end{itemize}}
\newcommand{\resumeItemListStart}{\begin{itemize}}
\newcommand{\resumeItemListEnd}{\end{itemize}\vspace{-5pt}}

%-------------------------------------------
%%%%%%  RESUME STARTS HERE  %%%%%%%%%%%%%%%%%%%%%%%%%%%%


\begin{document}

%----------HEADING----------
% \begin{tabular*}{\textwidth}{l@{\extracolsep{\fill}}r}
%   \textbf{\href{http://sourabhbajaj.com/}{\Large Sourabh Bajaj}} & Email : \href{mailto:sourabh@sourabhbajaj.com}{sourabh@sourabhbajaj.com}\\
%   \href{http://sourabhbajaj.com/}{http://www.sourabhbajaj.com} & Mobile : +1-123-456-7890 \\
% \end{tabular*}

\begin{center}
    \textbf{\Huge \scshape Hudson Liu} \\ \vspace{1pt}
\small 443-882-5497 $|$ \href{https://hudson.is-a.dev}{\underline{hudson.is-a.dev}} $|$ \href{https://github.com/Hudson-Liu}{\underline{github.com/Hudson-Liu}} $|$ hudsonliu0@gmail.com 
\end{center}


%-----------EDUCATION-----------
\section{Education}
  \resumeSubHeadingListStart
    \resumeSubheading
      {Gilman School}{Baltimore, MD}
      {High School Diploma}{Aug. 2021 -- Jun. 2025}
    \resumeSubheading
      {Johns Hopkins University}{Baltimore, MD}
      {Visiting Student, Future Scholars}{Aug. 2024 -- Jun. 2025}
    \resumeSubheading
      {Commuity College of Baltimore County}{Baltimore, MD}
      {Associate's in Computer Science}{Aug. 2020 -- Jun. 2025}
  \resumeSubHeadingListEnd

%-----------EXPERIENCE-----------
\section{Experience}
  \resumeSubHeadingListStart

    \resumeSubheading
      {Intern, Solo Developer of MISST Project | Sleep Staging w/ ResNets}{June 2020 -- Present}
      {Johns Hopkins University School of Medicine}{Baltimore, MD}
      \resumeItemListStart
      \resumeItem{Created an end-to-end data pipeline w/ Keras}
      \resumeItem{Developed an }
      %include links lil bro
      \resumeItem{Presented as first author at 7th Annual "Johns Hopkins Sleep \& Circadian Research Day" Symposium}

      \resumeItemListEnd

    \resumeSubheading
      {ASPIRE Intern | Image Synthesis of Microstructures w/ DDPMs}{June 2020 -- Present}
      {JHU Applied Physics Laboratory}{Laurel, MD}
      \resumeItemListStart
      \resumeItem{Created a data pipeline for simulating grain evolution from pairs of processing parameter via JAX-AM's PFM module. Used Python \& Bash scripts.}
      \resumeItem{Used aforementioned data pipeline for investigating processing-parameter-to-microstructure connections for Inconel-718 undergoing LPBF-AM solidification processes.}
        \resumeItem{Developed a DDPM-based model (Diff-PFM: Diffusion Probabilistic Field Model) employing the data pipeline, alongside MICRESS and ThermoCalc.}
        \resumeItem{Trained Diff-PFM on high-performance computing (HPC) cluster using distributed training on 6 H100 GPUs.}
        \resumeItem{Published paper on Diff-PFM as second author in Journal of "Metallography, Microstructure \& Analysis", DOI: \href{https://doi.org/10.1007/s13632-024-01130-w}{\underline{doi.org/10.1007/s13632-024-01130-w}}}
        \resumeItem{Presented twice as sole author at Aspire Student Showcase.}
        \resumeItem{Diff-PFM was presented @ APL AI Symposium \& Integrated Computational Materials and Enginnering (ICME) for Defense conference.}
        \resumeItem{APL News published an article highlighting Diff-PFM.}
      \resumeItemListEnd
    \resumeSubheading
    {Team Member | Robot Design \& Coding}{June 2020 -- Present}
      {FIRST Tech Challenge, Team DeJava 11695}{Baltimore, MD}
      \resumeItemListStart
      \resumeItem{Developed a CNN for object detection of game objects from robot's vision sensor.}
      \resumeItem{Solo designer of robot's 3-axis lift mechanism. Incorporated a turntable mechanism, with 3 drawer slides rigged in a cascading fashion.}
      \resumeItem{Worked on a sensor fusion system for aggregating neural network predictions with data from ultrasonic sensors and odometers.}
      \resumeItem{Programmed the mechanum chassis' drive code, along with the autonomous portion of robot movement.}
      \resumeItem{Volunteered to teach low-income inner city elementary students about principles of mechanical engineering (via the Gilman Bridges program).}
      \resumeItem{Demoed robot to Gilman Middle School students, as part of a collaborative outreach initiative. Helped guide multiple teams in the FIRST Lego League competition.}
      \resumeItem{Team was awarded the Control Award, Design Award, and Connect Award}
      \resumeItem{Winning alliance of CHS-MD Laurel Qualifier}
      \resumeItemListEnd

    \resumeSubheading
      {Team Member | ML Developer}{June 2020 -- Present}
      {Kaggle Happywhale Competition}{Baltimore, MD}
      \resumeItemListStart
      \resumeItem{Used OpenCV for contour mapping}
      \resumeItemListEnd

  \resumeSubHeadingListEnd


%-----------PROJECTS-----------
\section{Projects}
    \resumeSubHeadingListStart
      \resumeProjectHeading
          {\textbf{RCM Layer} $|$ \emph{Python, TensorFlow, Keras, Matplotlib, Sphinx}}{June 2020 -- Present}
          \resumeItemListStart
            \resumeItem{Created a novel architecture, RCM (Recurrent Complete Multidigraph), outperforming dense layers}
            \resumeItem{Developed a Keras implementation of RCMs as a layer}
            \resumeItem{Wrote script for generating 10-Gigapixel diagram of trained RCM for MNIST}
            \resumeItem{Designed a 3D Model of RCM for K6 in Fusion 360}
            \resumeItem{Auto-generating documentation via ReadTheDocs & Sphinx, hosted on GitHub Pages}
          \resumeItemListEnd
      \resumeProjectHeading
          {\textbf{++C Esolang (PostC)} $|$ \emph{C++}}{May 2018 -- May 2020}
          \resumeItemListStart
             \resumeItem{Created a new esolang, ++C: a postfix-based esolang based on C++ syntax}
            \resumeItem{Wrote ++C article on esolang wiki}
          \resumeItemListEnd
        \resumeProjectHeading
          {\textbf{Fleat} $|$ \emph{Python, Sphinx}}{July 2022 -- Aug. 2022}
          \resumeItemListStart
          \resumeItem{Created a Python library, Fleat (\bold{F}ast \bold{LEA}rning rate \bold{T}uner)}
          \resumeItem{Employed a 2D CNN for predicting ideal learning rates from images of NN architectures}
            \resumeItem{Auto-generating documentation via ReadTheDocs & Sphinx, hosted on GitHub Pages}
          \resumeItemListEnd
          
    \resumeSubHeadingListEnd

\section{Activities/Extracurriculars}
    \resumeSubHeadingListStart
      \resumeProjectHeading
        {\textbf{JV Cross Country/JV Indoor Track/JV Outdoor Track, Gilman School}}{June 2020 -- Present}
      \resumeProjectHeading
        {\textbf{2nd Chair Alto Saxohpone, Peabody Wind Orchestra}}{May 2018 -- May 2020}
      \resumeProjectHeading
            {\textbf{Co-Founder \& Co-President of AI Club, Gilman School}}{May 2018 -- May 2020}
      \resumeProjectHeading
            {\textbf{Phi Theta Kappa Honors Society, CCBC}}{May 2018 -- May 2020}
    \resumeSubHeadingListEnd

%
%-----------PROGRAMMING SKILLS-----------
\section{Technical Skills}
 \begin{itemize}[leftmargin=0.15in, label={}]
    \small{\item{
     \textbf{Languages}{: Python, Java, C/C++, HTML/CSS, Lua} \\
     \textbf{Developer Tools}{: Git, Anaconda, Docker, Neovim, Arch Linux} \\
     \textbf{Libraries}{: Keras, PyTorch, TensorFlow, Pandas, NumPy, Matplotlib, DearPyGUI}
    }}
 \end{itemize}


%-------------------------------------------
\end{document}
